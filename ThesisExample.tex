\documentclass[english]{IFIletter}

\usepackage[utf8]{inputenc}
\usepackage{babel}
\usepackage{graphicx}
\usepackage{float}
\usepackage{listings}
\usepackage{color}
\usepackage{hyperref}

\begin{document}

\address{Prof.\ Dr.\ Michael B\"ohlen}
\position{Professor}
\IFIgroup{Database Technology}
\IFIemail{boehlen}
\IFIphone{5 43 33}
\signature{Prof.\ Dr.\ Michael B\"ohlen\\\position}

\toname{}
\toaddressstreet{}
\toaddresscity{}
\toaddresscountry{}

\begin{letter}

\date{\today}

\opening{\textbf{MSc Thesis \\ Implementing Learned Cardinality
    Estimation in a Database Systems Context}}

\vspace{0.5cm}

Query optimization in database management systems heavily relies on
the cardinalities of result relations.  For instance, to choose
between different query plans the database system must accurately
estimate the cardinalities of intermediate results.  Cardinality
estimation in most database systems is based on the assumption that
attributes are independent, which rarely holds in reality and easily
leads to estimations that are orders of magnitude off.

In order to provide more accurate estimations, recent efforts have
been trying to apply learned models to replace the traditional
cardinality estimation component in database systems.  For example,
Kipf et al.~\cite{DBLP:conf/cidr/KipfKRLBK19} proposed to apply
multi-set convolutional neural networks to estimate the cardinalities
whereas Sun et al.~\cite{DBLP:conf/sigmod/Sun0021} proposed to use
segmentation to get a better estimation.  Although these research
works have shown their advantages, they have not been integrated into
existing database systems and have not been evaluated in production
environments.

The goal of this master thesis is to implement a learned cardinality
estimation method and integrate it into, e.g., the PostgreSQL database
system.  After the integration, an evaluation between the learned
cardinality estimation and the built-in estimation inside PostgreSQL
should be performed.

\textbf{Tasks}

\begin{enumerate}\itemsep=15pt

\item \textbf{Task 1: Literature review and prerequisite study}
  \begin{itemize}
  \item Study the relevant research work on learned cardinality
    estimation, including but not limited to
    \cite{DBLP:conf/edbt/HayekS20}, \cite{DBLP:conf/cidr/KipfKRLBK19},
    \cite{DBLP:conf/sigmod/Sun0021}, and
    \cite{DBLP:journals/pvldb/WangQWWZ21}.
  \end{itemize}

\item \textbf{Task 2: Implement cardinality estimation with base
    tables only}
  \begin{itemize}
  \item Based on the content of base tables, implement the learned
    cardinality estimation approach with linear regression and
    possibly other machine learning techniques, e.g., neural networks.
  \item The implemented approach should only depend on the content of
    the table.  Thus, the estimation is based on the data and not on
    any queries.
  \item Test the results with selected queries, for example:
     \begin{verbatim}
     	SELECT x, y FROM test WHERE x<11 AND y<21;
     \end{verbatim}
   \vspace{-15pt}
   where table \texttt{test} is populated by
     \begin{verbatim}
     	INSERT INTO test 
     	SELECT generate_series(1,100), generate_series(11,110);
    \end{verbatim}
   \vspace{-15pt}
 \end{itemize}

\item \textbf{Task 3: Implement cardinality estimation with workload}
  \begin{itemize}
  \item Based on workload information, implement the learned
    cardinality estimation described by Hilprecht and Binnig
    \cite{DBLP:journals/corr/abs-2105-00642} for selection and join
    queries, such as:
      \begin{verbatim}
      	SELECT x, y, m FROM test INNER JOIN test2 on x>m and y<m;
      \end{verbatim}
    \vspace{-15pt}
  \item In contrast to Task 2, learned cardinality estimation with
    workload information shall also learn from workload information,
    i.e., the current query, previous queries and possibly system
    resources.  Thus, the implemented approach shall not only utilise
    the content of the table but also the queries and other useful
    information.
  \end{itemize}

\item \textbf{Task 4: System Implementation}
  \begin{itemize}
  \item Investigate the feasibility of integrating learned cardinality
    estimation into PostgreSQL or another suitable system.
  \item Integrate the learned cardinality estimation with base tables
    that was implemented in the Task 2 into a system context.
  \item Integrate the learned cardinality estimation with workload
    information that was implemented in the Task 3 into a system
    context.
  \end{itemize}

\item \textbf{Task 5: Evaluate different approaches on synthetic and
    real-world datasets}
  \begin{itemize}
  \item Evaluate different cardinality estimation approaches with
    synthetic datasets, e.g., the dataset generated with
    \texttt{generate\_series(1, 100)}.
  \item Use four different real-world datasets (Census, Forest, Power
    \cite{Dua:2019} and DMV\cite{NysTransport}) to evaluate the
    learned cardinality estimation with the built-in estimation
    technique in PostgreSQL.
  \item Analyse the training time of learned cardinality estimation
    and the running time of different cardinality estimation
    techniques. Identify their advantages, disadvantages and
    limitations.
  \end{itemize}

\item \textbf{Task 6: Write and defend the thesis}
  \begin{itemize}
  \item Describe the implementations, results and evaluations in your
    Master's thesis.
  \item Present and defend your Master's thesis in the DBTG group
    meeting.
  \end{itemize}
\end{enumerate}

\vspace{0.5cm}

\bibliographystyle{plainurl}
\bibliography{refs}

\vspace{1cm}

\textbf{Supervisors}:
\begin{itemize}
\item Prof. Dr. Michael B\"ohlen (boehlen@ifi.uzh.ch)
\item Prof. Dr. Anton Dignös (adignoes@ifi.uzh.ch)
\item Qing Chen (qing@ifi.uzh.ch)
\end{itemize}

\vspace{0.5cm}
\textbf{Start Date}: July 15th, 2021

\vspace{0.5cm}
\textbf{End Date}: January 14th, 2022

\vspace{2cm}
\closing{University of Zurich\\Department of Informatics}

\end{letter}

\end{document}
