\documentclass[]{article}

%opening
\title{Notes}
\author{Andrin Rehmann}

\begin{document}

\maketitle

\section{Datapoints of supercomputers}

\begin{center}
	\begin{tabular}{ c c c c }
		& Piz Daint GPU \cite{piz_daint} & Eiger & Summit \\ 
		\hline
		Number of Nodes & 5704 & & \\
		Mem. Cap. / Node & 64 GB & cell5 & cell6 \\   
		CPU Model & Intel Xeon E5-2690 v3 \\
		CPU Mem. Bandwith  & 68 GB/s \\
		GPU Model & NVIDIA Tesla P100 \\
		GPU Mem. Cap. & 16 GB \\
		GPU Mem. Bandwith & 732 GB/s \\
		GPU Single Prec. Perf. & 9.3 tFlops
	\end{tabular}
\end{center}

\subsection{Terminology}

\begin{enumerate}
	\item Memory Bandwith: read / write speed of semiconductor memory
	\item 

\end{enumerate}

\section{Analysis}


We assume one particle requires 32 bytes of storage. Where the XYZ coordinates require $4\times3 = 12$ bytes of storage, the rest is used for additional information.

\subsection{Piz Daint} 
We can store $10^9$ particles in the memory of Piz Daint. 
\bibliographystyle{plain}
\bibliography{reference}

\end{document}
