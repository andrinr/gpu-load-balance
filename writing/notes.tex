\documentclass[]{article}

%opening
\title{Notes}
\author{Andrin Rehmann}

\begin{document}

\maketitle

\section{Datapoints of supercomputers}

\begin{center}
	\begin{tabular}{ c c c }
		& Piz Daint GPU \cite{piz_daint} & Summit \\ 
		\hline
		Number of Nodes & 5704 & 4608\\
		CPU Mem. Cap. & 64 GB & 512 GB  \\   
		CPU Model & Intel Xeon E5-2690 v3 & IBM POWER9 \\
		CPU Mem. Bandwith  & 68 GB/s \\
		CPU Number of Cores & 12 \\
		GPU Model & NVIDIA Tesla P100 & NVIDIA Volta V100s \\
		GPU Mem. Cap. & 16 GB & 96 GB\\
		GPU Mem. Bandwith & 732 GB/s \\
		GPU Interconnect Bandwidth & 32 GB/s \\
		GPU Single Prec. Perf. & 9.3 tFlops
	\end{tabular}
\end{center}

\subsection{Terminology}

\begin{enumerate}
	\item Memory Bandwidth: read / write speed of semiconductor memory
	\item 

\end{enumerate}

\section{Analysis}

\begin{itemize}
	\item 
	One particle requires 32 bytes of storage. Where the XYZ coordinates require $4\times3 = 12$ bytes of storage, the rest is used for additional information.
	
	\item
	A single iteration of the binary cut algorithm requires $32 \times N$ operations.
	
	\item 
	All particles are assumed to be stored in the CPU memory initially.
	
	\item
	We perform the analysis on one billion particles. This means we need 32 GB storage for all the particles and 12 GB for only the XYZ positions.
\end{itemize}


\subsection{Piz Daint} 
When executing the algorithm on the CPU, for a single iteration of the binary cut algorithm we perform a scan over an axis of all particle positions. This translates to $10^9 \times 4 bytes = 4 GB$.
On the CPU this results in an execution time of 
\begin{center}
	$32 \times \frac{ 4 GB}{68 GB/s} = 1.8 s$ 
\end{center}

If we do the same on the GPU we get an execution time of:
\begin{center}
	$32 \times \frac{4 GB}{732 GB/s} + 2 \times \frac{4 GB}{32 GB/s} = 0.38 s$ 
\end{center}



\bibliographystyle{plain}
\bibliography{reference}

\end{document}
